\section{Roteiro de Execução}

\subsection{Malha de COV’s}

Caso na etapa de avaliação preliminar, não se conseguir 
desenvolver um modelo conceitual consistente e detalhado da 
área que identifique todas as fontes primárias de 
contaminação (históricas e atuais) associadas à operação do 
empreendimento, o responsável técnico deverá executar a 
investigação da ocorrência de COV’s no solo. O objetivo da 
malha de investigação de COV’s terá natureza orientativa 
para definição dos locais onde serão executadas as 
sondagens de poços de monitoramento, sendo considerada um 
mecanismo de auxílio na investigação da presença de 
contaminação.

O mapeamento das concentrações de COV’s deverá ser 
realizado a partir de uma malha com espaçamento regular 
(5x5m), considerando sempre, a distribuição dos 
equipamentos e dutos em operação ou desativados.

Define-se a malha considerando as áreas comprometidas com 
instalações. Para as demais áreas do empreendimento 
(estacionamento, pátio), com área total de até 10.000 m2 (1 
ha), o espaçamento da malha de COV’s deverá ser de 10m, e 
de 20m para empreendimentos com metragens superiores.

Sempre que observados indícios de contaminação no solo, 
deverá ocorrer o adensamento da malha para melhor 
caracterização, ainda nesta fase de avaliação.

Com objetivo de estabelecer os critérios mínimos de 
controle de qualidade da realização da investigação de 
COV’s no solo, deverão ser observados, minimamente, os 
seguintes itens:

• Realizar em campo a aferição do medidor de vapores antes 
do início das medições, visando estabelecer o branco da 
área.

• Nunca realizar medições de COV’s durante a descarga de 
combustíveis.

• Nunca realizar medições próximas às bombas no momento do 
abastecimento de automóveis.

• Fazer medições a meio metro de profundidade. Em casos de 
presença de piso impermeabilizado, a leitura deverá ser 
realizada assim que a camada de piso seja ultrapassado..

• Utilizar haste de 3/4 in de diâmetro para aquisição das 
medições nas perfurações e nunca realizar estas medições 
diretamente no furo e sondagem.

• Estabelecer um tempo padrão para aquisição da medição dos 
COV’s nas perfurações, o qual deverá ser definido pelo 
responsável técnico em função do Modelo Conceitual Inicial.

• Anexar ao relatório, o certificado de calibração do 
equipamento de medição de COV’s na faixa de medição para 
BTEX e HPA’s.

• Preencher cuidadosamente o furo com calda de cimento é 
tarefa obrigatória e visa evitar a passagem de efluentes 
contaminados.

