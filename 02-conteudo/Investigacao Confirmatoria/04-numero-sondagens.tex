\subsection{Determinação do número de sondagens}

A determinação do número de sondagens a serem executadas 
para coleta de amostras de solo e instalação de poços de 
monitoramento deverá ter como base o Modelo Conceitual 
Inicial (MCI), desenvolvido na fase de Avaliação 
Preliminar, bem como os resultados de investigação obtidos 
pela implantação da malha de COV’s.

A escolha dos pontos de sondagem para amostragem de solo 
será balizada pelos hot spots identificados no mapa de 
isoconcentrações de COV’s resultante das medições de campo. 
Na ausência de pontos anômalos, as sondagens executadas 
para instalação de poços de monitoramento e avaliação de 
contaminação em fase dissolvida, deverão ser locadas à 
jusante das fontes primárias de contaminação. Mesmo com a 
presença de hot spots identificados na malha de COV's, 
todas as áreas fontes de contaminação identificadas 
anteriormente devem ser avaliadas.

Caso as sondagens não tenham atingido o nível freático, as 
sondagens para avaliação de contaminação em solo deverão 
estar localizadas o mais próximo possível da potencial 
fonte primária de contaminação.

Em situações nas quais a malha de COV’s não evidencie a 
presença de hot spots, e que as sondagens de solo não 
tenham atingido o aquífero freático, deverão ser executadas 
sondagens o mais próximo possível das potenciais fontes 
primárias de contaminação. Para os pontos comprometidos com 
as instalações, deverão ser executados furos com leituras 
às seguintes profundidades: para tanques, a 4,0m e para 
base de bombas, filtros, CSAO, base de respiros, áreas de 
troca de óleo e lubrificação, área de lavagem de veículos e 
local de armazenamento óleo usado, a 1,50m.

Ressalta-se que as justificativas de impossibilidade de 
penetração no terreno, em caso de uso de equipamentos 
inadequados, não serão consideradas. A recomendação é que 
se execute a perfuração com equipamento mecanizado, até o 
atingimento do nível de água freático, aprofundando pelo 
menos 2 m na camada aquífera. A sondagem poderá ser 
interrompida quando for atingido o topo rochoso, ou 
mediante justificativa técnica do profissional técnico 
responsável. Em caso de topo rochoso, pelo menos três 
sondagens deverão ser realizadas para avaliação da 
continuidade do mesmo. No caso da presença de topo rochoso, 
a presença do mesmo deverá ser justificada por meio de 
informações geológicas como a presença de rocha alterada e 
fragmentos de rocha ao longo do perfil de sondagem, bem 
como da presença de afloramentos de rocha próximos da área 
avaliada.

