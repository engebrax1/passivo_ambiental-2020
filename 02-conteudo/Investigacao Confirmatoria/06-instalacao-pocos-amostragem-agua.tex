\subsection{Instalação dos poços de amostragem de água}

Os pontos de sondagem para amostragem de solos também 
poderão servir à instalação de poços de amostragem de água, 
que deverão ser mantidos como poços de monitoramento (PM’s).

A preexistência de poços de monitoramento no empreendimento 
não desobriga a realização da malha de COV’s e instalação 
de novos PM’s para amostragem de água, se necessário. Os 
perfis construtivos dos poços de monitoramento deverão 
constar do relatório final.

Nas amostras de água deverão ser determinados os seguintes 
parâmetros: BTXE (benzeno, tolueno, xilenos e etilbenzeno), 
HPA’s (hidrocarbonetos poliaromáticos) e TPH’s 
(hidrocarbonetos totais de petróleo).

Recomenda-se que a profundidade final dos poços de 
amostragem de água seja de no mínimo 2,0m abaixo do nível 
d’água, construídos segundo ABNT NBR 15.495-1 e 15.495-2.

Deverão ser mantidos a título de “Poços de Monitoramento” - 
PM’s, os poços de amostragem de água instalados por ocasião 
dos estudos de Investigação Confirmatória. Para tanto, 
deverão ser instalados em locais adequados e protegidos de 
infiltração de efluentes, acúmulo de águas pluviais e/ou de 
eventuais danos provocados pela passagem de veículos. Em 
casos de avaria na estrutura dos poços de monitoramento, 
medidas de reparo imediatas devem ser adotadas.

Os poços instalados serão úteis tanto para o simples 
monitoramento da integridade da água do aqüífero freático 
durante a vigência da licença ambiental, quanto no 
monitoramento do site em processos de remediação 
implantados.

A nomenclatura dos poços de monitoramento adotadas em 
estudos anteriores deve ser mantida, a fim de facilitar a 
rastreabilidade dos poços e das análises realizadas ao 
longo do tempo.

Caso exista a possibilidade de migração vertical dos 
contaminantes, deverá ficar a critério do responsável 
técnico, a instalação de poços de monitoramento com seção 
filtrante afogada, construídos conforme definem as normas 
técnicas, a fim de avaliar a contaminação e evitar a 
migração vertical da pluma de contaminação.

