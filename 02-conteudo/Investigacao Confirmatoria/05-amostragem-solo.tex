\subsection{Amostragem de solo}

Para aquisição de amostras de solo para realização de 
análises químicas laboratoriais, deverão ser adotadas as 
recomendações descritas nas normas ABNT NBR 16.434 - 
Amostragem de Resíduos Sólidos, Solos e Sedimentos - 
Análise de Compostos Orgânicos Voláteis (COV’s) e ABNT NBR 
16.435 - Controle da Qualidade na Amostragem para fins de 
Investigação de Áreas Contaminadas.

Adicionalmente ao descrito nas normativas da ABNT, deverão 
ser observados os seguintes itens durante a execução das 
sondagens:

• Não utilizar fluido de perfuração, bem como emprego de 
graxas ou outro material.

• Realizar a limpeza de todos os equipamentos utilizados 
antes do início de cada perfuração, sendo obrigatória a 
utilização de detergente neutro e não fosfatado, água 
corrente e enxague final com água destilada e deionizada.

• Apresentar documentação fotográfica de todo o processo de 
amostragem.

• Elaborar o perfil descritivo do material identificado 
para cada sondagem de solo executada.

• Apresentar a descrição dos equipamentos para execução das 
sondagens e amostragem de solo.

• Identificar cada frasco com os dados correspondentes ao 
ponto amostrado.

• Georreferenciar todos os pontos de amostragem, informando 
as coordenadas UTM, cota e o Datum utilizado.

• Apresentar cadeia de custódia e ficha de recebimento de 
amostras pelo laboratório.

Deverá ser elaborado Plano de Amostragem desenvolvido com 
base nos resultados da etapa de Avaliação Preliminar 
considerando o MCI, no qual devem constar todas as 
justificativas quanto à escolha da metodologia de 
perfuração, número e localização das sondagens, poços de 
monitoramento, tipo e profundidade de amostragem, dentre 
outros. Atenção especial deverá ser dada às profundidades a 
serem atingidas e intervalos a serem amostrados nas áreas: 
de tancagem, descarga, abastecimento, das caixas 
separadoras, de disposição de óleo usado, dentre outras.

Nas amostras de solo deverão ser determinados os seguintes 
parâmetros: BTEX (benzeno, tolueno, xilenos e etilbenzeno), 
HPA’s (hidrocarbonetos poliaromáticos) e TPH’s 
(hidrocarbonetos totais de petróleo).

Os laudos analíticos das amostras de solo, águas 
subterrâneas e outros materiais avaliados deverão estar de 
acordo com o definido na ABNT NBR ISO/IEC 17025, devendo 
necessariamente ser identificados o local onde foi coletada 
a amostra (nome e endereço), o ponto de amostragem, as 
datas em que as amostras foram coletadas, a extração e a 
análise que foram realizadas, os métodos analíticos 
adotados, os fatores de diluição, os limites de 
quantificação, os resultados do branco de laboratório, da 
recuperação de traçadores (surrogate) e da recuperação de 
amostra padrão. É importante que os laudos laboratoriais 
fornecidos apresentem mecanismos de verificação da 
autenticidade dos mesmos após a emissão. Os laudos deverão 
ser acompanhados da ficha de recebimento de amostras (check 
list) emitida pelo laboratório no ato de recebimento das 
amostras e da cadeia de custódia referente às amostras 
coletadas, devidamente preenchidas e assinadas.

O laboratório selecionado para análise deve possuir 
obrigatoriamente ABNT NBR ISO/IEC 17.025 e Certificado de 
Cadastramento de Laboratório - CCL, conferido pelo 
Instituto Água e Terra. A relação atualizada de 
laboratórios encontra-se disponível no site do órgão 
ambiental.

