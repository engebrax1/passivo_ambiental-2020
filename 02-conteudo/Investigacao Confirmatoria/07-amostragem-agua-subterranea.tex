\subsection{Amostragem de água subterrânea}

Para a coleta de água subterrânea, será exigido um Plano de 
Amostragem, com justificativa da escolha do local de 
implantação do poço de monitoramento. Cada passo 
constituinte da presente etapa deverá ser ilustrado por 
imagens fotográficas.

Após coletadas, receberão acondicionamento em frascos 
apropriados, devendo ser imediatamente armazenadas em 
cooler a 4ºC, observando-se os prazos para realização das 
análises.

As amostras de água subterrânea deverão ser coletadas 
conforme orientações das normas ABNT NBR 15.847 e ABNT NBR 
16.435. É recomendada a adoção do método de amostragem por 
purga de baixa-vazão, após transcorrido o prazo de 01 (um) 
ano a partir da publicação desta resolução. A adoção de 
outros métodos de amostragem ficará a critério do 
responsável técnico, mediante justificativa técnica.

Transcorrido o prazo de 02 (dois) anos da publicação desta 
resolução, a coleta de amostras deverá ser executada por 
profissional certificado junto ao INMETRO - ABNT NBR 
ISO/IEC 17025:2017, independente do método de amostragem 
utilizado.

Durante o procedimento de coleta por baixa vazão deverão 
ser monitorados diversos parâmetros físico-químicos, com a 
purga sendo concluída após a estabilidade hidrogeoquímica 
avaliada pela determinação dos parâmetros previstos no 
quadro abaixo.