\subsection{Histórico}

%Levantamento histórico do uso da área e entorno, com 
%auxílio de imagens multitemporais, entrevistas com 
%moradores locais e, obrigatoriamente, pesquisa junto a 
%instituições (Prefeitura, Corpo de Bombeiro, IPEM e outros).
%
%Coleta de dados existentes sobre o histórico operacional do 
%empreendimento (eventos de vazamentos, reformas, troca de 
%bandeiras, mudança de layouts, autuações, alteração da 
%volumetria da tancagem, levantamento de antigos ensaios de 
%estanqueidade, dentre outros). Indispensável que se proceda 
%à consulta do processo no órgão ambiental, caso exista.
%
%A caracterização do entorno deverá ser realizada em um raio 
%de 200 (duzentos) metros a partir do perímetro do 
%empreendimento, com a identificação de receptores 
%potenciais de ingestão da água subterrânea (poços cacimba, 
%poços tubulares), de locais onde foram ou são desenvolvidas 
%atividades com potencial de contaminação e de áreas com 
%contaminação comprovada.
%
%Ao término desta atividade, deverá ser elaborado texto 
%explicativo com resumo das características do entorno do 
%empreendimento e planta em escala apropriada, contendo:

\subsubsection[Uso e ocupação do solo]{Uso e ocupação do solo, com a identificação de 
receptores potenciais ou bens a proteger, como por exemplo, 
áreas residenciais, áreas comerciais, áreas industriais, 
áreas de lazer, áreas de produção agropecuária, 
piscicultura, hortas, escolas, hospitais, creches, etc}

\subsubsection{A localização e a classificação dos corpos d’água 
superficiais e subterrâneos}

\subsubsection{A localização de poços de abastecimento cadastrados junto 
ao Instituto das Águas do Paraná e SANEPAR num raio de 500 
m do entorno do empreendimento.}

\subsubsection{A localização de poços rebaixamento, drenos, fontes, 
nascentes e todos os tipos de poços de abastecimento não 
cadastrados no Instituto das Águas e na SANEPAR.}

\subsubsection{A localização de Área com Potencial de Contaminação (AP), 
Área com Suspeita de Contaminação (AS), Área Contaminada 
(AC), Área em Processo de Monitoramento (AM) e Área 
Reabilitada para Uso Declarado (AR) eventualmente 
existentes na região considerada.}

\subsubsection{A indicação da existência de rede de esgoto, de água 
tratada, de águas pluviais e de outras utilidades 
subterrâneas.}
