\section{Elaboração de Modelo Conceitual Inicial (MCI)}

Representar a situação da área quanto à possível 
contaminação existente e sua relação com o entorno, 
incluindo bens a proteger. Será utilizado como base para o 
planejamento das etapas posteriores de investigação, com 
atualizações a cada fase de avanço do conhecimento e 
consolidação previstas nas etapas de detalhamento e de 
reabilitação da área, quando necessárias.

Área fonte de contaminação está relacionada a um 
determinado processo operacional que pode ocasionar uma 
contaminação. No Modelo Conceitual Inicial deverão ser 
identificadas todas as áreas fontes de contaminação 
avaliadas, como área de tancagem, área da pista de 
abastecimento, área de lavagem de veículos, área de troca 
de óleo, área da CSAO, dentre outras.

Nas áreas fontes, deverão ser identificadas todas as fontes 
primárias de contaminação, como tubos de descarga à 
distância, os tanques, linhas de sucção de combustível, 
unidades de abastecimento, sistemas de filtragem de diesel, 
CSAO, sistema de drenagem oleosa, base de respiros, local 
de armazenamento de óleo usado, sistema de troca de óleo, 
sumidouro/fossa séptica e local de lavagem de veículos.

As áreas fontes devem ser categorizadas em áreas potenciais 
de contaminação ou em áreas suspeitas de contaminação, 
quando houver evidências, indícios, ou fatos que permitam 
suspeitar da existência de contaminação. Por fim, a área 
como um todo deverá ser classificada entre Área Potencial 
de Contaminação e Área Suspeita de Contaminação, dependendo 
das informações levantadas no Modelo Conceitual Inicial.