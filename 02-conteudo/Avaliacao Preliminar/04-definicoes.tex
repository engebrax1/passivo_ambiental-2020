\chapter{DEFINIÇÕES}

\begin{description}
	\item[Água Subterrânea] águas que ocorrem naturalmente 
	no subsolo.
	
	\item[Área classificada] área na qual uma atmosfera 
	explosiva de gás está presente ou é provável sua 
	ocorrência a ponto de exigir precauções especiais para 
	construção e utilização de equipamentos elétricos.
	
	\item[Área com potencial de contaminação] aquela onde 
	estão sendo ou foram desenvolvidas atividades 
	potencialmente contaminadoras, isto é, atividades onde 
	ocorre ou ocorreu o manejo de substâncias cujas 
	características físico-químicas, biológicas e 
	toxicológicas podem causar danos e/ou riscos aos bens a 
	proteger.
	
	\item[Área comprometida com as instalações] local que 
	efetivamente abriga ou abrigou instalações de linhas, 
	tanques, bombas, filtros, caixas separadoras, base de 
	respiro, armazenamento de óleo usado e lubrificação e 
	troca de óleo.
	
	\item[Atmosfera explosiva] mistura com ar, sob 
	condições atmosféricas, de substâncias inflamáveis na 
	forma de gás, vapor, névoa e substâncias combustíveis, 
	na qual, após a ignição a combustão se propaga através 
	da mistura não consumida.
	
	\item[Contaminação] introdução nos recursos ambientais 
	de agentes patogênicos, de substâncias tóxicas ou 
	radioativas, ou de outros elementos em concentrações 
	nocivas ao ser humano, à fauna e à flora.
	
	\item[COV’s] Compostos Orgânicos Voláteis presentes em 
	solos contaminados por hidrocarbonetos constituintes de 
	combustíveis.
	
	\item[Franja capilar] faixa de água subsuperficial 
	mantida por capilaridade acima da zona saturada.
	
	\item[Líquidos inflamáveis] líquidos que possuem ponto 
	de fulgor inferior a 37,8 ºC e pressão de vapor menor 
	ou igual a 275,6 kPa (2068,6 mmHg), denominados Classe 
	I.
	
	\item[Passivo ambiental] toda poluição, degradação ou 
	contaminação sofrida pelo meio ambiente resultante de 
	atividade poluidora ou de sua desativação.
	
	\item[Solo] sistema aberto, dinâmico, sujeito a fluxos 
	internos e externos, onde ocorrem processos físicos, 
	químicos e biológicos, resultantes da alteração e 
	evolução do material original (rocha ou mesmo outro 
	solo) pela ação de organismos vivos, clima, influência 
	do relevo e tempo de exposição.
	
	\item[TPH] Hidrocarbonetos totais de petróleo.
	
	\item[TPH total] Quantidade mensurável de 
	hidrocarbonetos totais de petróleo presentes na matriz 
	ambiental analisada (soma dos hidrocarbonetos 
	resolvidos com a mistura complexa não resolvida).
	
	\item[TPH-Resolvido] Quantificação dos hidrocarbonetos 
	que são separados na análise cromatográfica, 
	apresentando picos bem definidos. São todos os picos 
	presentes no cromatograma.
	
	\item[MNCR (Mistura Complexa Não-Resolvida)] 
	Quantificação da Mistura de hidrocarbonetos de petróleo 
	que não são separados na análise cromatográfica, 
	causando uma elevação na linha de base do cromatograma.
	
	\item[TPH Fingerprint] Quantificação de hidrocarbonetos 
	totais de petróleo de C10 a C40.
	
	\item[Cromatograma] Gráfico da concentração do analito 
	versus o tempo/volume de eluição.
	
	\item[Elementos Notáveis] equipamentos ou instalações 
	auxiliares existentes na área de empreendimentos 
	contemplados no Art. 02 desta Resolução, que servem ao 
	armazenamento, à distribuição de combustíveis líquidos, 
	ao acúmulo de resíduos ou tratamento de efluentes 
	líquidos.
\end{description}