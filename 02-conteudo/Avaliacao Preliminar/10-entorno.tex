\subsection{Caracterização da ocupação do entorno do 
empreendimento}

• Por se tratarem de atividades consideradas potencialmente 
poluidoras e geradoras de acidentes ambientais (Resolução 
nº 273, de 29 de novembro de 2000), o órgão ambiental 
considera que todo o empreendimento que armazena, revende 
ou distribui combustíveis líquidos, deva ser considerado 
como Classe III (ABNT NBR 13.786, ou outra que venha 
sucedê-la).

A classe do posto de combustível é definida pela análise do ambiente em torno do posto de serviço, numa distância de 100 m a partir do seu perímetro. Identificado o fator de agravamento no ambiente entorno, o posto de serviço deve ser classificado no nível mais alto, mesmo que haja apenas um fator desta classe. 

Essa análise permite a seleção dos equipamentos e sistemas a serem utilizados para o SASC.  As classes estão divididas em quatro níveis, numerados de 0 a 3.

Segundo a Resolução SEMA 32/2016, a caracterização do entorno segundo a NBR 13.786/2001 da ABNT, que regulamenta a classificação ambiental de postos de serviço define que: "empreendimentos localizados em área urbana, assim definida por Lei Municipal, serão considerados classe 3, conforme disposto na Lei Estadual nº 14.984/05".

Na \cref{tab:nbr13786} consta a classificação do empreendimento, classificado como de classe 3.


\setlength\LTleft{0pt}
\setlength\LTright{0pt}

\begin{longtable}{@{\extracolsep{\fill}}@{\hspace{1cm}}p{12.7cm}C{1.3cm}}
	\caption{Caracterização do entorno segundo a NBR 13.786/2001}
	\label{tab:nbr13786}\\
	
	\toprule
	\bfseries &  \bfseries Sim/Não \\\midrule
	
	\endfirsthead
	
	\caption{Caracterização do entorno segundo a NBR 13.786/2001 (Continuação)}\\
	
	\toprule
	
	\bfseries &  \bfseries Sim/Não \\\midrule
	
	\endhead
	
	\midrule
	
	\multicolumn{2}{r}{\textit{\footnotesize Continua na próxima página}} \\ 
	
	\endfoot
	
	\multicolumn{2}{l}{Classificação do posto de serviço:}\\	
	\multicolumn{2}{l}{\hfill \small{0} \CustomSquare \hfill \small{1} \CustomSquare \hfill \small{2} \CustomSquare \hfill \small{3} \CustomCheckedBox}\\
	
	\bottomrule
	
	\multicolumn{2}{p{15cm}}{\scriptsize \textsuperscript{*} Entende-se como atividade e operações de risco o armazenamento e manuseio de explosivos, bem como locais de carga e descarga de inflamáveis líquidos (base e terminal)}\\\addlinespace
	
	\multicolumn{2}{l}{text}
	

	\endlastfoot
	
	
	
	\multicolumn{2}{l}{\bfseries Classe 0}\\\cmidrule(r{9cm}){1-1}
	Não possui nenhum dos fatores de agravamento das classes seguintes \dotfill &\CustomSquare \\
	
	\multicolumn{2}{l}{\bfseries Classe 1}\\\cmidrule(r{9cm}){1-1}
	Rede de drenagem de águas pluviais \dotfill & \CustomCheckedBox \\	
	Rede subterrânea de serviços (água, esgoto, telefone, energia elétrica, etc.) \dotfill & \CustomCheckedBox \\	
	Fossa em áreas urbanas \dotfill & \CustomSquare \\
	Edifício multifamiliar com até quatro andares \dotfill & \CustomSquare \\
	
	\multicolumn{2}{l}{\bfseries Classe 2}\\\cmidrule(r{9cm}){1-1}
	Asilo \dotfill & \CustomSquare \\
	Creche \dotfill & \CustomSquare \\
	Edifício multifamiliar com mais de quatro andares \dotfill & \CustomSquare \\
	Favela em cota igual ou superior a do posto \dotfill & \CustomSquare \\
	Edifício de escritórios comerciais com quatro ou mais pavimentos \dotfill & \CustomSquare \\
	Poço de água, artesiano ou não para consumo doméstico \dotfill & \CustomSquare \\
	Casa de espetáculos ou templo \dotfill & \CustomSquare \\
	Escola \dotfill & \CustomSquare \\
	Hospital \dotfill & \CustomSquare \\
	
	\multicolumn{2}{l}{\bfseries Classe 3}\\\cmidrule(r{9cm}){1-1}
	
	Favela em cota inferior a do posto \dotfill & \CustomSquare \\
	Metrô em cota inferior a do solo \dotfill & \CustomSquare \\
	Garagem residencial ou comercial construída em cota inferior a do solo  \dotfill & \CustomSquare \\
	Túnel construído em cota inferior a do solo \dotfill & \CustomSquare \\
	Edificação residencial, comercial ou industrial construída em cota inferior a do solo 
	\dotfill & \CustomSquare \\
	Atividades industriais e operações de risco\textsuperscript{*} \dotfill & \CustomSquare \\
	Água do subsolo utilizada para abastecimento público da cidade (independente do perímetro de 100 metros) \dotfill & \CustomCheckedBox \\
	Corpos naturais superficiais de água à:&\\
	\hspace{0.5cm} Abastecimento doméstico \dotfill & \CustomSquare \\
	\hspace{0.5cm} Proteção de comunidade aquáticas \dotfill & \CustomSquare \\
	\hspace{0.5cm} Recreação de contato primário (natação, esqui aquático e mergulho) \dotfill & \CustomSquare \\
	\hspace{0.5cm} Irrigação \dotfill & \CustomSquare \\
	\hspace{0.5cm} \parbox{12cm}{Criação natural e ou intensiva de espécies destinadas á alimentação humana (CONAMA nº 20)\dotfill} \dotfill & \CustomSquare \\
	\midrule
\end{longtable}