\chapter{AVALIAÇÃO PRELIMINAR DE PASSIVOS AMBIENTAIS}

\section{Objetivo}

Tem como objetivo principal constatar evidências, indícios 
ou fatos que permitem suspeitar da existência de 
contaminação na área sob avaliação, por meio de 
levantamento de informações disponíveis sobre o uso atual e 
pretérito da área (Gerenciamento de Áreas Contaminadas - 
GAC), de modo a subsidiar o desenvolvimento das próximas 
etapas de investigação.

Durante a etapa de avaliação preliminar podem ser obtidas 
evidências (ocorrência de fase livre, identificação de 
ambiente confinados com risco de explosão, dentre outros) 
que indiquem a necessidade de adoção de medidas 
emergenciais visando a proteção da saúde humana e de outros 
bens a proteger.